% General Paper Template created by Adam Green
% Last revised 3/2/18

\documentclass[12pt]{article}

%%%%%%%%%%%%%%%%%%%%%%%%%%%%
% Standard Packages
%%%%%%%%%%%%%%%%%%%%%%%%%%%%
\usepackage{amsmath,amsfonts,amsthm,amssymb}
\usepackage[margin = 2cm]{geometry}


%%%%%%%%%%%%%%%%%%%%%%%%%%%%
% Graphical Packages
%%%%%%%%%%%%%%%%%%%%%%%%%%%%
\usepackage{graphicx}
\usepackage{subfig}
\usepackage{float}
\usepackage{tikz}
\usepackage{wrapfig}
% Syntax: \begin{wrapfig}[lineheight]{position}{width}

\usepackage{tikz}
\usetikzlibrary{shapes, arrows}
%\usepackage{pgfplots}

\graphicspath{{Figures/}} % set directory for figures

%%%%%%%%%%%%%%%%%%%%%%%%%%%%
% Colors
%%%%%%%%%%%%%%%%%%%%%%%%%%%%
\usepackage{xcolor}

\definecolor{myblue1}{RGB}{76, 142, 185}
\definecolor{myblue2}{RGB}{25, 100, 126}
\definecolor{myblue3}{RGB}{41, 110, 180}

\definecolor{mygreen1}{RGB}{88,165,87}
\definecolor{mygreen2}{RGB}{91,165,98}

\definecolor{myred1}{RGB}{221, 28, 26}
\definecolor{mypurple}{RGB}{122,48,108}


%%%%%%%%%%%%%%%%%%%%%%%%%%%%
% Table and Array Packages
%%%%%%%%%%%%%%%%%%%%%%%%%%%%
\usepackage{tabu}
\usepackage{booktabs}


%%%%%%%%%%%%%%%%%%%%%%%%%%%%
% Citation Packages
%%%%%%%%%%%%%%%%%%%%%%%%%%%%
\usepackage{hyperref}
\hypersetup{allcolors = myblue1,
	allbordercolors = myblue1, 
	filecolor = myblue1, 
	linkbordercolor = myblue1,
	urlbordercolor = white
}

\usepackage{cleveref}
\crefname{equation}{Eqn.}{Eqns.}
\crefname{figure}{Fig.}{Figs.}
\crefname{table}{Tab.}{Tabs.}

%\usepackage{caption}
%	\captionsetup{format = , justification= }

%%%%%%%%%%%%%%%%%%%%%%%%%%%%
% Text Formatting packages
%%%%%%%%%%%%%%%%%%%%%%%%%%%%
\usepackage{multicol}
\usepackage{parskip} 
\usepackage{indentfirst} 
\setlength{\parindent}{0.75cm}
\usepackage{fancyhdr}
%	\pagestyle{fancy}
\usepackage{enumerate}
\usepackage{wrapfig}
\usepackage{lipsum}
\usepackage{xhfill}
\usepackage{comment}
\usepackage{verbatim}


%%%%%%%%%%%%%%%%%%%%%%%%%%%%
% Custom Commands
%%%%%%%%%%%%%%%%%%%%%%%%%%%%
% Red marks to get attention
\newcommand{\red}[1]{\textbf{\textcolor{myred1}{#1}}} % Red Text
\newcommand{\redmark}{\textcolor{myred1}{\rule{4mm}{4mm} }} % Red Dash
\newcommand{\redline}{\noindent\xhrulefill{myred1}{3pt}} % Red Rule


% Lowercase Captial Letters
\newcommand{\scap}[1]{\textsc{\MakeLowercase{#1}}} % Makes caps small so it doesnt SHOUT


% Math Commands: first and second order partials, Laplacian
\newcommand{\evaluate}{\Bigr\rvert}
\newcommand{\ppd}[1]{\frac{\partial}{\partial#1}}
\newcommand{\ppsd}[1]{\frac{\partial^2}{\partial #1^2}}
\newcommand{\ppnd}[2]{\frac{\partial #1}{\partial #2}}
\newcommand{\ppsnd}[2]{\frac{\partial^2 #1}{\partial #2^2}}
\newcommand{\lap}{\nabla^2}

% Quantum bra- ket- commands
\newcommand{\bra}[1]{\langle #1 |}
\newcommand{\ket}[1]{| #1 \rangle}
\newcommand{\bracket}[2]{\langle #1 | #2 \rangle}

% Redfine equation and figure references to include "Eqn. ()" and "Fig. _"
\newcommand{\figref}[1]{Fig.\ \ref{#1}}
%\let\originaleqref=\eqref
\renewcommand{\eqref}{Eqn.\ \originaleqref}


% Custom Matrix Spacing
% Syntax: \begin{matrix}[scale]
\makeatletter
\renewcommand*\env@matrix[1][\arraystretch]{%
	\edef\arraystretch{#1}%
	\hskip -\arraycolsep
	\let\@ifnextchar\new@ifnextchar
	\array{*\c@MaxMatrixCols c}}
\makeatother


% Email Commands: Taken from Flip
% Syntax: \email{email}
\newcommand{\email}[1]{\href{mailto:#1}{\textcolor{mygreen1}{#1}}}

% Institution Environment: Taken From Flip
\newenvironment{institutions}[1][2em]{\begin{list}{}{\setlength\leftmargin{#1}\setlength\rightmargin{#1}}\item[]}{\end{list}}

% Link to External Webpage
% Syntax: \link{url}
\newcommand{\link}[1]{\href{#1}{\textcolor{mygreen1}{\texttt{#1}}}}



\begin{document}
The general iterator for Simpson's rule is:	
$$ I(a,b,N) = \sum_i^N \frac{\Delta x}{3} \left[  f(x_i) + 4f(x_i+\Delta x) + f(x_i+2\Delta x) \right] $$
where, after each step, we increment by $2\Delta x$. This is because if we incremented by only $\Delta x$, we would be over-sampling our function, or in other words, our ``rectangles'' would overlap each other.

To make this look like Newman's form, we fix the integration between $x\in [a,b]$ and, for illustrative purposes, choose $N=4$ steps, $b$ is then fixed to be $b=a+8\Delta x$ since we are incrementing by $2\Delta x$.

Explicitly writing out the summation yields:
$$ I(a,b,4) = I_1(a,b,4) + I_2(a,b,4) + I_3(a,b,4) + I_4 (a,b,4) $$ where each $I_i$ is the $i\text{th}$ term in the summation. Writing out each $I_i$ yields:
$$I_1= \frac{\Delta x}{3} \left[  f(a) + 4f(a+\Delta x) + f(a+2\Delta x) \right]$$

$$I_2= \frac{\Delta x}{3} \left[  f(a+2\Delta x) + 4f(a+3\Delta x) + f(a+4\Delta x) \right] $$

$$I_3= \frac{\Delta x}{3} \left[  f(a+4\Delta x) + 4f(a+5\Delta x) + f(a+6\Delta x) \right] $$

$$I_4= \frac{\Delta x}{3} \left[  f(a+6\Delta x) + 4f(a+7\Delta x) + f(b) \right] $$
where in the last summation, $a+8\Delta x = b$ by definition. The middle term in each $I_i$ contains an odd prefactor attached to $\Delta x$ which stops at $7 = N-1$. The first term in the $I_4$ summation has $6=N-2$ attached to $\Delta x$. The values of these prefactors will ultimately fix where the summation ends.

Adding all the $I_i$ terms together, we can rewrite the summation as:
$$ I(a,b,N) \sim \frac{\Delta x}{3} \left[ f(a) + f(b) + \sum_n 4 f[a+(2n-1)\Delta x] + 2f[a+2n\Delta x] \right] $$
where the prefactor $2n+1$ is always odd, and $2n$ is always even. However, the stopping point of the summation over n depends whether $n$ is even or odd. We can make this dependence explicit and write the summation in the form of Newman's equation:
$$ I(a,b,N) = \frac{\Delta x}{3} \left[ f(a) + f(b) + \sum_{n,\text{ odd}}^{N-1} 4 f[a+n\Delta x] + \sum_{n,\text{ even}}^{N-2} 2f[a+n\Delta x] \right] $$

	
	
	
\end{document}